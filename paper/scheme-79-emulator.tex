\documentclass[12pt]{article}
\usepackage{graphicx}
\usepackage{listings}
\usepackage{color}

\author{Sammy Furr}
\title{CSMC 305 Final Project: Emulating the SCHEME-79 Chip}
\date{\today}

\graphicspath{ {./images/} }

\definecolor{sgreen}{rgb}{0.52,0.53,0}
\definecolor{gray}{rgb}{0.5,0.5,0.5}
\definecolor{scyan}{rgb}{0.16,0.63,0.60}

\lstset{frame=tb,
  language=lisp,
  aboveskip=3mm,
  belowskip=3mm,
  showstringspaces=false,
  columns=flexible,
  basicstyle={\small\ttfamily},
  numbers=none,
  numberstyle=\tiny\color{gray},
  commentstyle=\color{gray},
  stringstyle=\color{scyan},
  breaklines=true,
  breakatwhitespace=true,
  tabsize=3
}

\begin{document}
\begin{titlepage}
\maketitle
\end{titlepage}
\section{Introduction}
I first got interested in LISP when I started to use Emacs as my
editor at the beginning of college.  I was shocked when I learned how
old the language is---it predates almost all languages still used
today, and was certainly a high-level language decades before
C\cite{lisp}.  Dialects of LISP were being developed while we were
still figuring out a lot about computing that we take for granted
today.  In the late 70's, as Guy L. Steele Jr. and Gerald Jay Sussman
were creating SCHEME at the MIT Artificial Intelligence Laboratory,
swaths of researchers were hard at work trying to fit millions of
transistors into chips the size of a fingernail, using new Very Large
Scale Integration (VLSI) design techniques.  In 1979, these two worlds
would join as part of the MPC79 project\cite{mpc79}.  Sussman and Steele, along
with Jack Holloway and Alan Bell, would create a chip that directly
interpreted SCHEME.\cite{s79}\par
I was struggling to find a LISP machine to emulate for my final
project.  The papers I found described stack machines or PDP clones
with some hardware instructions to do the LISP-y functions (cons, car,
etc.) thrown in.  I was beginning to sour on my idea when I finally
stumbled across the SCHEME-79 paper. A chip that directly interpreted
s-expressions, the basis of one of the least imperative languages of
all time?  A processor that was entirely specified in LISP and then
``compiled into artwork''\cite[12]{s79}?  Uniting software and
hardware abstraction techniques, the functional and the imperative,
s-expressions and silicon, the SCHEME-79 chip is a remarkable piece of
computing history.  I hope that when you run the emulator and see
s-expressions moving through registers you will be as amazed as me!
\section{How the Chip Works}
The SCHEME-79 chip is remarkably simple.  It contains
\begin{enumerate}
  \item 10 special-purpose registers.
  \item 3 Programmable Logic Arrays (PLAs), which combine to form a single Finite
    State Machine (FSM).
  \item 44 pads for connecting to the outside world.
\end{enumerate}
This is all connected by a 32 bit bus.  That's really
it---s-expressions are directly stored in the registers, each of which
contains a type field.  Each 7-bit type is the equivalent of an opcode
on a traditional processor.  On each instruction cycle the FSM sees a
type, such as type 005\textsubscript{8}-procedure, transitions it's state
accordingly, and executes the micro-word that goes with the type.
\begin{lstlisting}
  (deftype procedure
    (assign *val* (&cons (fetch *exp*) (fetch *display*)))
    (&set-type *val* closure)
    (dispatch-on-stack))
\end{lstlisting}
Procedure takes the memory address of the procedure definition in the
expression register, conses it to the environment stored in the
display register, and sticks all this in the value register to make a
closure.  This closure will be applied and evaluated by previous
types, which have been stored on the stack.  To transition to the next
state, the FSM dispatches the type in the stack register.  Computation
proceeds like this, recursively evaluating s-expressions until the
program is over, at which point the processor halts.
\section{Emulating SCHEME-79}
The SCHEME-79 paper comes with a full specification of the microcode
on the chip.  Emulating the chip is not as simple, of course, as
copying the microcode verbatim into a file and interpreting.  The real
challenge in emulating the chip is writing functions to emulate the
functionality of the chip that is implemented directly in hardware,
not programmed onto the PLA.  These functions are not described in the
microcode specification---only named.  Doing this involved a fair
amount of trial-and-error, and a lot of re-reading the paper for
hints.  Luckily, some of the microcode is fairly self-documenting.
Some of it, however, is decidedly not:
\begin{lstlisting}
  ...(eval-exp-popj-to internal-apply))  ;Amazing! Where did retpc go?
\end{lstlisting}
``Amazing! Where did retpc go?'' is the original authors' comment, not
mine.\cite[39]{s79}\par
As it turns out eval-exp-popj-to is a function in hardware.  It stores
the microcode return-address (type), internal-apply in this example,
in the type field of the stack register, saves the stack register onto
the stack, and then evaluates the expression in the exp register. This
is where retpc, our return program counter, went.  Here's the code to
emulate this behavior:
\begin{lstlisting}
  (define (eval-exp-popj-to type)
    (&set-type *stack* type)
    (save *stack*)
    (dispatch-on-exp-allowing-interrupts))
\end{lstlisting}\par
In the end, the emulator includes a few basic components, just like
the chip it emulates:
\begin{enumerate}
\item Vectors to hold the types, which are used like addresses of
  micro-words.
\item Functions and macros to bind micro-code addresses to the
  microcode given in the paper.
\item A variety of direct hardware functions and macros not given in
  the paper.
\item A vector and some functions to emulate memory.
\item A function to step through individual instruction cycles,
  running a program.
\end{enumerate}
\section{TODO}
\subsection{Garbage Collection}
The emulator is still lacking one large piece of functionality from
the original chip---garbage collection.  A large portion of the paper
is dedicated to garbage collection, and you can see why when running
the emulator.  The recursive nature of SCHEME means that values are
constantly being thrown on the stack.  The stack is implemented on the
heap as a LISP list, which is beautifully simple and also incredibly
quick to fill memory.  The garbage collector is fully specified, but I
didn't have time to completely understand the algorithm for it, and
decided to forgo implementing it to focus on simply getting the
interpreter running.
\subsection{Compilation}
Programs for the original chip were written in LISP, and then compiled
to the S-code that SCHEME-79 actually runs.  Though S-code is still
recognizable as LISP, it differs in how some operations are connected
together, and is more explicit.  The paper doesn't come with a
compiler, so for now S-code has to be hand-written.  For example, the
ultra-simple function application:
\begin{lstlisting}
  ((lambda () 3))
\end{lstlisting}
Translates into the following S-code:
\begin{lstlisting}
ADDR CAR           . CDR
1001 01100001002 . 00000000000 ;; apply-no-args to addr 1002
1002 00500002000 . 00000000000 ;; procedure at addr 2000
2000 10000000003 . 00000000000 ;; self-evaluating-immediate 3
\end{lstlisting}
All numbers are specified in octal.\par
This is both tedious and difficult to write by hand, and a compiler is
definitely needed!  I still need a fuller understanding of every
aspect of the chip in order to write a compiler, however.  Barring
writing a compiler, the functionality to read S-code from a file is
needed, as you must currently type S-code into the file that houses
code for the emulator---sorry!
\section{Conclusion}
Emulating SCHEME-79 was a challenge, but a completely worthwhile one
for such an awesome piece of engineering.  The chip is truly
extraordinary, and the way it was implemented was revolutionary for
the time.  In the era of the all powerful general-purpose processor a
special-purpose chip so influenced by one type of programming language
may seem like an antiquity.  Looking closely at the features of a
modern x86 or ARM processor reveals that this couldn't be further from
the truth.  When you load a webpage using AES, your connection is
encrypted using hardware directly on your chip.  The need for fast
encryption and decryption has become so ubiquitous that every modern
consumer processor includes instructions to perform \textit{a specific
  encryption algorithm}.  Just like the need to quickly interpret
SCHEME drove the SCHEME-79's hardware development, the need to quickly
decrypt and encrypt webpages led chip-makers to develop direct
hardware for the task.  We don't make so many standalone
single-purpose processors now. Instead, we put them in every
general-purpose processor we make.\par
The techniques of abstraction used to design the SCHEME-79 chip enable
the vast chips we have today.  With a minimum feature width of 5
microns\cite[21]{s79}, SCHEME-79 was on the bleeding edge.  Standing
in a classroom you can see how far we've come.  The classroom is the
gate width on SCHEME-79.  Hold up a pinky.  That's the width of a
gate doing the AES cryptography on Intel's latest 10nm
node\cite{wiki}.  Just incredible.

\pagebreak
\bibliographystyle{acm}
\bibliography{scheme-79-emulator.bib}
\pagebreak
\end{document}